%########################################################
% skeletons fuer latex2e documente
% AUSFUEHRLICHERE Version: skeletonAusfuehrlich.tex (dec04)
%########################################################
\documentclass[12pt,a4paper]{scrartcl}  
%\documentclass[twocolumn,showpacs,preprintnumbers,amsmath,amssymb]{revtex4}
%\documentclass[preprint,showpacs,preprintnumbers,amsmath,amssymb]{revtex4}
%\documentclass[a4paper]{foils}
     % Praesentation a la "Powerpoint", vgl. ~/prosperPraesentation/          
%\documentclass[tu,colorBG,slideColor,pdf]{prosper} 

\usepackage{graphicx} % definiert \includegraphics[width=???mmm]{eps Bild}
\usepackage[dvips]{color}     %Definiert \definecolor und 
                              % red,yellow,green,blue,black,gray,white
\usepackage{lscape} % provides landscape environment fuer txt rotieren:
%\begin{landscape}.. \end{landscape}

%\usepackage{german}  %Fuck all german options do not work any more (9/2015)
                      % need defs.tex or defsSkript.tex to correct/circumvent;
                      % defsSubmit.tex does not have it (since English)
\usepackage{units} %\unit[3]{m/s^2} => 3 m/s^2 etc!
\usepackage{amsfonts,amsmath,amssymb} % crucial for correct setting of
                                % vectors etc

\usepackage{eurosym}  %Euro-Symbol: \euro{Zahl} oder euro{}

\usepackage{hyperref} 
%                  Internet-Link: \href{http://www.WasAuchImmer.html}
%                  {\blue{\underline{TextDesLinks} }}
%                  Lokaler Link: \hyperlink{targetName}{TextDesLinks}
%                  Link-Target: \hypertarget{targetName}
%                  Lokaler File-Link: \href{file:///home/ ...}
%  Info: \myHyperlink{http..}{Linktext}

\usepackage{cite} % z.B. Refs 1,2,3,5 werden zu [1-3,5]  zusammengefasst

%\textheight220mm
%\textwidth150mm
%\oddsidemargin0mm
%\topmargin0mm
%\parindent0mm


%\input{defs}          % includes auch ``colors.st''
\input{defsSkript}          % includes auch ``colors.st''
%\input{defsSubmit}          % includes auch ``colors.st''

\renewcommand{\textfraction}{0.2}  % Figures: At least 20% text at each page
\setlength{\parindent}{0.9em} %horizontal indentation first line
\setlength{\parskip}{0.5em}  %addtl. vertical space between paragraphs

\usepackage{psfrag}




%########################################################

\begin{document}
\title{Latex skeleton}
\author{Treibi}
\author{\large\sf\textbf{Martin Treiber}}
\date{\normalsize\sf\textbf{\today}}
%\date{\normalsize\sf\textbf{32.01.2019}} % if ``timestamp'' fixed

\maketitle % sonst wird nix gemacht!


% oder eigene Titelseite:
\thispagestyle{empty}
\begin{center}
\LARGE\sf\textbf{Titel}
\\[5mm]
\large\sf\textbf{Martin Treiber}
\\
\large\sf\textbf{\today}
\end{center}

%Inhaltsverzeichnis (selbe Schriftart wie \section, \mysection etc, 
% aber normalgross)

\tableofcontents

\section{Figures}
%###########################################################
\begin{figure}
%\fig{0.7\textwidth}{ASM_TrivialModel2sketch.eps}
\caption{\label{fig:sketchGen}Sketch of the traffic state estimation
  problem}
\end{figure}
%###########################################################

\section{Text und Formeln}

{\large\sf\textbf{Serifenlose gro\3e fette Schrift nur so:
Makro  $\backslash\!\!$ bfblack!}}

\section{Hyperlinks}
Nur im pdf!


Link auf 
\myHyperlink{http://www.mtreiber.de/movie3d/index.html}{www.traffic-simulation.de/movie3d}
mit {\tt $\backslash\!\!\!$ myHyperlink}.

Link auf
\href{http://www.mtreiber.de/movie3d/index.html}{www.traffic-simulation.de/movie3d}
mit {\tt $\backslash\!\!\!$ href} direkt.

Link auf \myHyperlink{http://www.traffic-simulation.de/index.html}
{www.traffic-simulation.de}.

Lokaler Link auf
\myHyperlink{file:///home/vwitme/staff/treiber/talks/LatexBeamerClass/Applet2d/Ramp/Ramp.html}{Onramp-Java-Simulation}.


%##########################################################
% References
%##########################################################

%(1) Erstellen

% bibstyles (voller Pfad noetig!): /home/vwitme/staff/treiber/tex/inputs/*.bst
% bib-Datenbanken (voller Pfad noetig!): /home/vwitme/data/books/bibtex/bibtex/*.bib

%\bibliographystyle{/home/vwitme/staff/treiber/tex/inputs/prsty} 
\bibliographystyle{/home/vwitme/staff/treiber/tex/inputs/alphadin}

%\bibliography{refs,/home/vwitme/data/books/bibtex/bibtex/database}  %TU
%\bibliography{refs,databaseLoc}  %Notebook

%(2a) Wenn fertig:

%\include{references.tex}

%(2b) oder (besser!) direktes Einbinden .bbl File

%(3) explizit:
\begin{thebibliography}{10}
\bibitem{Einstein-SRT}
A. Einstein, {\it Zur Elektrodynamik bewegter K\"orper}, 
Annalen der Physik {\bf 17}, 891-921 (1905).
% siehe http://de.wikipedia.org/wiki/Albert_Einstein#Werk
\bibitem{Einstein-Emc2}
A. Einstein, {\it Ist die Tr\"agheit eines K\"orpers von seinem 
Energieinhalt abh\"angig?}, Annalen der Physik {\bf 18}, 639-641 (1905).

\end{thebibliography}

\end{document}
%########################################################

